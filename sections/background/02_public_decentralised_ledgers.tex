\subsection{Public De-centralised Ledgers}

\subsubsection{Introduction to Public De-centralised Ledgers}

A public de-centralised ledger (PDL) is a recent invention which endeavours to make data publicly accessible through a de-centralised system providing no single point of failure. A PDL provides transparency where traditional centralised systems fall short and allow any willing and able party to be a part of the network. Furthermore, a PDL provides no easy way for any network moderator or specific party to control or override the network without the consensus of a significant number of the network's members. Combined, this technology introduces an entirely new way of dealing with transactional data and, as implemented below, indeed any sort of data which has some lifecycle.

\subsubsection{Relevance}

% TODO: Unsure of the relevance of merkel trees for this application

% \subsubsection{Merkel Trees}

% Merkel Dag is the network of objects, pointing to each other using the hash of each object (Merkel)

% \subsubsection{IPFS}

\subsubsection{Blockchain}

The Blockchain is the underlying technology that is used by cryptocurrencies such as Bitcoin\footnote{\href{https://bitcoin.org/en/}{Bitcoin (https://bitcoin.org)}}.

\subsubsection{Ethereum}

TODO
