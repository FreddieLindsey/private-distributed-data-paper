\subsection{Design Considerations}

Since this project is intended as a proof of concept as a forebearer to a future, more concrete solution, the design choices made are only relevant in the context of their communication and key theoretical objectives and achievements. This means that the choice of which blockchain technology to use, for example, is not strictly important since it only serves as a means to show the possibility of such a system. However, it is crucial to recognise the constraints of current technology and which of these are inherent through the concept of the technology not just a specific implementation.

The project aims to provide a de-centralised architecture (wherever possible) which is tamperproof and allows layered data access by default. De-centralised compute and storage platforms are therefore required. For the purposes of this project, Ethereum~\autocite{ethereum:2014:online} will be used for compute and IPFS~\autocite{ipfs:2017:online} will be used for storage. Ethereum provides a wrapper around the original blockchain technology used by cryptocurrencies such as Bitcoin~\autocite{bitcoin:2013:online}.
