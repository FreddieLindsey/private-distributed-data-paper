\section{Project Plan}

Over the course of the project, there will be several stages and for each of these stages an associated key milestone. Throughout this process, a mixture of research, discovery, implementation, and user testing must occur in order to give the project a well-rounded and well-evaluated outcome.

Below, I explain the desired timeline of the project and indicate when I intend to reach specific milestones.

\vspace{5 mm}
\begin{table}[!h]
  \centering
  \begin{tabularx}{0.9\textwidth}{ r | X }
    \textbf{10th February} & Complete Interim Report (Draft) \\
                           & At this point crucial research should have been undertaken to understand the viability of the project and to be able to assess it's preliminary chances of success. \\
                           [4ex]
    \textbf{17th February} & Prototype architecture \\
                           & Whilst working on preliminary architecture, the implementation of core features should have begun. By this point a more structured and well understood architecture should be written. \\
                           [4ex]
    \textbf{3rd March}     & Evaluate and consider second marker's comments
                           and advice \\
                           & After discussions with the second marker to the project, evaluate the points of concern and any advice on how to achieve the greatest success with the project. Move forward with this in mind and adjust any proposed plans or process as necessary. \\
                           [4ex]
    \textbf{5th March}     & First working prototype \\
                           & A basic working prototype must include the ability to read and write data at a basic level \\
                           [4ex]
    \textbf{19th June}     & Final Report Due \\
                           [4ex]
    \textbf{26th June}     & Submission of Project Archive
  \end{tabularx}
  \vspace{10 mm}
  \caption{
    Plan for project (given progress thus far)
  }
  \label{table:project_plan}
\end{table}

% PARTY A
% Private (unencrypted?) data
% Private key / Public key pair

% PARTY B
% Private (unencrypted?) data
% Private key / Public key pair

% PROXY
% Public keys for all users / institutions (verifies the identity of a party and allows re-encryption)
% Re-encryption keys for each API access (requires the 3rd party to read data)
% Encrypted data (using owner's private key - not readable by anyone else)

% No private keys nor unencrypted data is shared in the network
% Upon granting access, the user's device generates the re-encryption key locally (potentially multiple keys whereby the data is readable by anyone but the data is verified to be encrypted by a particular device??)
% This would require the use of proxy-reencryption such that any data uploaded by any device could be re-encrypted to be decrypted by any of the user's devices

% User passes private and encrypted data to proxy
% User sends the re-encryption key and access
% Access logs are encrypted with the user's public key as blobs on blockchain - only readable by the user themselves
% Who owns data - where do we store data that is written by party a but belongs to party b?

% What about if you used IPFS across public hardware - individuals etc. to create the world's biggest network of file storage stations, the mining being the storage of files, paid for by people using the network to store encrypted data. Filecoin?
