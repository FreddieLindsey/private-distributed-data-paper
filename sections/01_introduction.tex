\section{Introduction}

\begin{displayquote}{
  "\textbf{When it comes to control over our own data, health data must be where we draw the line.}"~\parencite{wilbankstopol:2016:article}
}\end{displayquote}

Over the last 350 years, the general public has proceeded, often unwittingly, to give up their right to privacy by exchanging personal data for the convenience of the modern world. One might attribute the origin of this movement in the UK to the introduction of paper money by the Bank of England~\parencite{bankofengland:2016:online} in 1694. This event heralded the idea of giving information about a person to an institution, be it a corporation or a government, in exchange for convenience. Before this time, one might have kept their savings 'under the mattress' and therefore there would be no sharing of one's wealth with another party. Whilst bank notes were introduced as a means to raise funds for a war, they also required the depositors (as a whole) to identify exactly how much money they had as a group. At this stage, this imposes no constraint on the depositor to give up any part of his unique identity, only form part of a wider, anonymous identity (the group).

% TODO: Swiss health identity card data

% - Privatisation of data
%   - Google, Facebook, NHS, Banks, retail stores (loyalty programs)
%   - Data Protection Act (limitations and corporate-focus)
% - User choice (who do I want to have my data and how)
% - Online identities
%   - Global identity tracking
%   - Conglomerate identity providers
% - Privatisation of data
%   - Google, Facebook, NHS, Banks, retail stores (loyalty programs)
%   - Data Protection Act (limitations and corporate-focus)
% - User choice (who do I want to have my data and how)
% - Online identities
%   - Global identity tracking
%   - Conglomerate identity providers
\subsection{Motivation}

Over the last 350 years, the general public has proceeded, often unwittingly, to give up their right to privacy by exchanging personal data for the convenience of the modern world, most often in the control of corporations and conglomerates.
\newline
One might attribute the origin of this movement in the UK to the introduction of paper money by the Bank of England~\parencite{bankofengland:2016:online} in 1694. The introduction of paper currency gave an opportunity to the Bank of England to start collecting data on the exchange of money nationally. At the time it is unlikely that any person exchanging gold for paper currency was aware of this signficant social change since they were more concerned with the convenience afforded to them. Before this time, a person might have kept their savings 'under the mattress' and almost certainly would not have shared information relating to their wealth with another party. Whilst bank notes were introduced as a means to raise funds for war, inadvertently the seed for a data revolution was sewn. Whilst no unique identities were shared at this point, they would be in years to come.

Fast forward several hundred years and we find ourselves in a society where it is commonplace to rely on few key corporations and organisations to control information nationally and internationally. As what is probably the world's most popular social network it is clear from Facebook's terms and conditions~\autocite{facebookterms:2015:online} that our use of the social network is subject to a few key conditions which restrict and change the status quo of our privacy as users. Foremost, it is apparent that whilst content posted on Facebook remains the property of the owner, Facebook has the right to use it how it wishes (as per the IP license) and hence is the controller of that data. One might choose to post it on Facebook, but one cannot stop Facebook using their content without removing it from all of Facebook's services and ensuring everyone with whom one has shared it with has also removed it from Facebook. Furthermore this allows Facebook to use any content posted for machine learning, training systems and providing commercial services using the intelligence gained from the distribution of content on the Facebook network. As someone who cares for their privacy, it is my opinion that this is not an acceptable status quo.

\begin{displayquote}{
  "\textbf{When it comes to control over our own data, health data must be where we draw the line.}"~\autocite{wilbankstopol:2016:article}
}\end{displayquote}



% TODO: Swiss health identity card data

% - Privatisation of data
%   - Google, Facebook, NHS, Banks, retail stores (loyalty programs)
%   - Data Protection Act (limitations and corporate-focus)
% - User choice (who do I want to have my data and how)
% - Online identities
%   - Global identity tracking
%   - Conglomerate identity providers


At the core of the motivation for this project lay several issues corresponding to the way in which society has been manipulated over time. It is my belief that we find ourselves in the current position without any ownership of our data because we've been keen (even greedy) as a society to reap the benefits of our data without considering the longer term security effects. We have dismissed the need to care and be responsible for our data. Below, I have highlighted the key domains in which we lack control that we should have over our personal data. Whilst written as a piece of fiction, we should be aware and concerned that ignoring the social issues with data transfer allows a world to form much similar to that of George Orwell's 1984~\autocite{orwell:1984:book} - we consider the likes of corporations synonymous with that of the 'Big Brother' character.

\subsubsection{Commoditisation of personal (and private) data}

There is no doubt that search tools such as those offered by Google and Microsoft, retail stores such as those offered by Amazon, and social networks such as Facebook and Twitter, dramatically enhance our lives and give us capabilities we would never have otherwise. Often as consumers we are eager to accept these benefits without considering the means by which they are offered to us.

% TODO: Freedom to use personal data
% \subsubsection{Freedom to use personal data}

