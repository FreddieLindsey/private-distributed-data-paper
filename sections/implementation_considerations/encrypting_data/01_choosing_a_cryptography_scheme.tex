\subsubsection{Choosing A Cryptography Scheme}

Fundamental to this project is the underlying end-to-end encryption of data. Without end-to-end encryption this project represents nothing more than a public ledger of the distribution of the files belonging to an identity, readable by any party. Henceforth, choosing a suitable and secure cryptography scheme is critical.

\paragraph{Initial Library Choice}

An asymmetric cryptography scheme with a public-private key pair is required in order to facilitate proxy re-encryption. As the most popular scheme~\footnote{As used in SSL encryption globally} RSA~\cite{rsa:1978:article} seemed a suitable candidate. Whilst the proxy re-encryption scheme authored by \cite{afgh:2006:article} meets many more desirable properties than that authored by \cite{ivandodis:2003:inproceedings}, the latter allows the use of different cryptography schemes to support proxy re-encryption. Whilst in a development stage, the priorities of this project lie in progressing implementation before proving security. With a popular library~\footnote{\href{https://www.npmjs.com/package/node-forge}{`node-forge'} had 1,091,913 downloads between 9th May 2017 and 9th June 2017} having a widely used RSA implementation, and with no other popular cryptography libraries available using other asymmetric cryptography schemes, RSA was chosen for a first attempt.

\paragraph{RSA Unsuitability}

Whilst originally thought that RSA would be a suitable scheme, it became apparent that this was not the case. Fundamentally, the scheme authored by \cite{ivandodis:2003:inproceedings} requires splitting an RSA private key into two halves and distributing these halves to a proxy and end-user separately. The re-encryption and decryption processes are in fact two separate decryptions with each half of the original private key capable of decrypting the original ciphertext.

An RSA private key is composed of elements some $e, d, n, p, q$ as solutions to the following equations, where $p, q$ are some large primes, $1 < e < \phi(n)$, and $\text{gcd}$ is some function that calculates the greatest common denominator of two integers.

$$
\begin{aligned}
  n &=& pq \\
  \phi(n) &=& (p - 1)(q - 1) \\
  \text{gcd}(e, \phi(n)) &=& 1 \\
  de & \equiv & 1 (\text{mod } \phi(n))
\end{aligned}
$$

The $e, d$ elements represent the public key and private key exponents respectively. Therefore, to split the private key, one must find two other numbers which when combined achieve the same exponent as $d$. Let's not make any assumptions about how these two numbers are used, other than that each represents a private key usable with the RSA decryption algorithm.

Given that any one of the two numbers we require is representative of an RSA private key, it must have some $d^\prime$ that is a function of $d$. Given this, it must be possible to calculate the two primes $p^\prime, q^\prime$ which compose $d^\prime$. However, the problem of determining such primes is that it is equivalent to finding two primes $a, b$ that compose $d$, i.e. cracking the original RSA encryption. Since this has been proven to be a hard problem, we can safely assume that splitting the RSA private key in this way is at least non-trivial and certainly impractical as a scheme.

% TODO: Write more about the trapdoor knapsack problem

\paragraph{ElGamal}

Whilst there is no popular ElGamal cryptography library available for JavaScript~\footnote{At the time of writing, only the `elgamal' library was \href{https://www.npmjs.com/package/elgamal}{available}. This library had only 42 downloads in the same period as `node-forge' and was assumed to be unsuitable.}, an implementation of an ElGamal-derived cryptography scheme extended for proxy re-encryption was available in Java, authored by \cite{nicscrypto:2012:online}. This library, `nics-crypto', implements the scheme authored by \cite{afgh:2006:article}. Whilst contradicting one of the project's objectives to be a completely client-side application, thereby only using JavaScript and JavaScript-transpilable languages, this library provides the necessary solution. It is discussed further below.
