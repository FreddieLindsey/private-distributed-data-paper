\subsubsection{Choosing A Cryptography Scheme}

Fundamental to this project is the underlying end-to-end encryption of data. Without end-to-end encryption this project represents nothing more than a public ledger of the distribution of the files belonging to an identity, readable by any party. Henceforth, choosing a suitable and secure cryptography scheme is critical.

An asymmetric cryptography scheme with a public-private key pair is required in order to facilitate proxy re-encryption. As the most popular scheme~\footnote{As used in SSL encryption globally} RSA~\cite{rsa:1978:article} seemed a suitable candidate. Whilst the proxy re-encryption scheme authored by \cite{afgh:2006:article} meets many more desirable properties than that authored by \cite{ivandodis:2003:inproceedings}, the latter allows the use of different cryptography schemes to support proxy re-encryption. Whilst in a development stage, the priorities of this project lie in progressing implementation before proving security.

Whilst originally thought that RSA would be a suitable scheme, it became apparent that this was not the case. Fundamentally, the scheme authored by \cite{ivandodis:2003:inproceedings} requires splitting an RSA private key into two halves and distributing these halves to a proxy and end-user separately. The re-encryption and decryption processes are in fact two separate decryptions with each half of the original private key capable of decrypting the original ciphertext.

Understanding that an RSA private key is composed of elements (including but not limited to) $d, n, p, q$ as solutions to the following equations:

$$
\begin{aligned}
  
\end{aligned}
$$
