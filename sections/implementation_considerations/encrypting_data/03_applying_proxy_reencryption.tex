\subsubsection{Applying Proxy Re-encryption}

Having established the (AFGH) scheme which should be used and how it is applied, the below examples explain the implementation of this scheme in this project. For clarity, an entity sharing content will be referred to as a delegator and an entity with whom content is being shared will be referred to as a delegatee.

\paragraph{Proxy Re-encryption Gateway}

As mentioned above, the proxy re-encryption library that is used by this project is written in Java. Such that we can interface with it from the JavaScript application interface, we require a RESTful API. Since the application is designed to be used in a decentralised manner, we assume that this library will be re-written or otherwise encapsulated into the application interface natively, but accept that for the purposes of this project every client is expected to run this gateway locally. Example interactions with this gateway are presented below, and represent function calls on the Encryption Packet class described.

\paragraph{Encryption Packet}

The Encryption Packet class is designed to be used once per instantiation. It holds input data and an AFGH-compatible key pair (ElGamal). This input data can either be a message that is to be encrypted or a ciphertext to be decrypted. Neither the private key nor public key half of the key pair is explicitly required by the class, such that it allows users without a private key (such as those encrypting a message for another party) or public key to still successfully create an instance. In either case, the appropriate methods will throw an exception if the packet is asked to perform an action for which it does not have appropriate arguments.

The class can be instantiated as follows:

\begin{figure}[H]
  \centering
  \begin{minted}{java}
	KeyPair keyPair = new KeyPair();
	byte[] data = ...;
	...
	EncryptionPacket encryptionPacket = new EncryptionPacket(keyPair, data);
  \end{minted}
  \caption{Creating an Encryption Packet}
  \label{code:encryption_packet_instantiation}
\end{figure}


Once the instance has been created, the packet should be either encrypted or decrypted:

\begin{listing}[H]
  \centering
  \begin{minted}{java}
EncryptionPacket encryptionPacket = new EncryptionPacket(keyPair, message);
...
byte[] firstLevelEncrypted  = encryptionPacket.getFirstLevelEncryption();
byte[] secondLevelEncrypted = encryptionPacket.getSecondLevelEncryption();
  \end{minted}
  \caption{
    Encrypting data with an EncryptionPacket
  }{
  	A message is encrypted (using either first or second level encryption) with the public key of the key pair.
  }
  \label{code:encryption_packet_encryption}
\end{listing}


\begin{listing}[H]
  \centering
  \begin{minted}{java}
EncryptionPacket encryptionPacket = new EncryptionPacket(keyPair, ciphertext);
...
byte[] firstLevelDecrypted  = encryptionPacket.getFirstLevelDecryption();
byte[] secondLevelDecrypted = encryptionPacket.getSecondLevelDecryption();
  \end{minted}
  \caption{
  	Decrypting data with an EncryptionPacket
  }{
  	A cipher text is decrypted (using either first or second level decryption) and the underlying message revealed.
  }
  \label{code:encryption_packet_decryption}
\end{listing}


\textit{Note: as per the \cite{afgh:2006:article} scheme, first and second level encryptions/decryptions are not cross-compatible. As shown above, only one decryption will be successful since the input ciphertext is either a first or second level encryption, but cannot be both.}

\paragraph{Encrypting data}

\begin{figure}[H]
  \centering
  \begin{minted}{json}
{
  "publicKey": "AbfqbOR+6y.../8yse7Km0=",
  "data": "SGVsbG8gV29ybGQNCg=="
}
  \end{minted}
  \caption{
  	JSON encrypt packet
  }{
  	A packet to encrypt the data 'Hello World' under the given public key.
  }
  \label{code:encrypt_data_json}
\end{figure}


When the user sends the above JSON to the API the following function is performed to encrypt the data.

\begin{figure}[H]
  \centering
  \begin{minted}{java}
	public byte[] getEncryption() {
		// Generate a symmetric key and symmetrically encrypt data
		bytes[] symmetricKey = encrypt();

		// Encrypt the symmetric key
		firstLevelEncrypted = AFGHProxyReEncryption.firstLevelEncryption(
			symmetricKey,
			publicKey,
		);

		// Return the result of concatenating the encrypted symmetric key
		// and the data encrypted with the symmetric key
		return mergeByteArrays(firstLevelEncrypted, symmetricEncryption);
	}
  \end{minted}
  \caption{Generating encrypted data}
  \label{code:encrypt_data}
\end{figure}


\paragraph{Decrypting data}

\begin{figure}[H]
  \centering
  \begin{minted}{json}
{
  "secretKey": "sUOtdGsmFOAMsaoLvbAc2icyG16U8/Zk2jaEW8Zw12A=",
  "data": "AAADBAAAAA...mh3wRInA=="
}
  \end{minted}
  \caption{
  	JSON decrypt packet
  }{
  	A packet to decrypt the data 'Hello World' from the ciphertext given the associated secret key.
  }
  \label{code:decrypt_data_json}
\end{figure}


When the user sends the above JSON to the API the following function is performed to decrypt the data.

\begin{figure}[H]
  \centering
  \begin{minted}{java}
public byte[] getDecryption() {
  // Split the input data into the encrypted key and encrypted data
  Tuple<bytes> dataPackets = splitPackets();

  // Decrypt the symmetric key
  bytes[] symmetricKey = asymmetricDecrypt(dataPackets.getFirst(), publicKey);

  // Decrypt the underlying data
  bytes[] message = symmetricDecrypt(dataPackets.getSecond(), symmetricKey);

  return message;
}
  \end{minted}
  \caption{
  	Decrypting message from ciphertext
  }{
  	A ciphertext is passed as data. The data is then arbitrarily decrypted and the underlying message returned.
  }
  \label{code:decrypt_data}
\end{figure}


\paragraph{Re-encryption of data}

\begin{figure}[H]
  \centering
  \begin{minted}{json}
	{
		"publicKey": "AZFDvizq24...i9k+JqN+0=",
		"data": "AXBWe7CDNt...AwsQoKWg=="
	}
  \end{minted}
  \caption{
  	JSON re-encrypt packet
  }{
  	A packet to re-encrypt a cipher text from $M_a$ to $M_b$ given a re-encryption key (shown as public key) $re_{a \rightarrow b}$.
  }
  \label{code:reencrypt_data_json}
\end{figure}


When the user sends the above JSON to the API the following function is performed to re-encrypt the data.

\begin{figure}[H]
  \centering
  \begin{minted}{java}
	public byte[] reencrypt() {
		// Split the input data into the encrypted key and encrypted data
		Tuple<bytes> dataPackets = splitPackets();

		// Reencrypt the symmetric key using the given re-encryption key
		bytes[] encryptedKey = asymmetricReencrypt(dataPackets.getFirst(), reencryptionKey);

		// Generate ciphertext from the encrypted symmetric key
		// and the data encrypted with the symmetric key
		bytes[] ciphertext = mergeByteArrays(encryptedKey, symmetricKeyData.getSecond());
        
		return ciphertext;
	}
  \end{minted}
  \caption{Reencrypting data}
  \label{code:reencrypt_data}
\end{figure}


